\chapter{Elaboration}

The principle of elaboration is this: Whatever you are studying in a course, try to expand (elaborate) upon what you are given. For example, if the instructor gives you 5 functions of the liver during a lecture, go out and try to find information about more functions (that the instructor didn't mention or perhaps doesn't know). In the end, you may add two more functions of the liver and in the process remember all of them better. For another example, if your assigned textbook discusses three theories of personality, try to find other theories in library books, other references, or even from talking to instructors other than your own. Don't limit yourself to what any one instructor (or textbook) tells you. They all have their biases which can prevent you from seeing (and learning) all the facts. When you elaborate, you end up understanding the information better and it will therefore be easier to recall and use.

So in summary, elaboration is a process whereby the learner expands upon the information given to them during a lecture, lab, reading assignment, etc. It's an act of empowerment, and puts the ultimate control of what is learned into the learner's hands. Right where the power should be in the first place! If a student were to practice the elaboration principle over a long period of time (and over many courses), he or she would surely ``look different'' to any potential employer or interviewer when a job was being sought out. This student would have a richer bank of information to call upon during the interview process (compared to other students). The other students with the same degree would ``look'' very similar to each other because they took mostly the same courses and received all the same information (because they didn't go out and elaborate). So set yourself apart (by learning more than the herd average) and the effort will be well rewarded.
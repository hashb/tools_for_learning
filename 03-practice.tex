\chapter{Distributed Practice}

Distributed practice is a technique whereby the student distributes his/her study effort in a given course over many study sessions that are relatively short in duration. This can be compared to massed practice (otherwise known as cramming) whereby the student conducts few but long study sessions for a given course. It has been proved beyond a shadow of a doubt that meaningful learning is promoted when distributed practice is conducted. In contrast, massed practice promotes rote learning. For the long-term benefit of the student, distributed practice should be the method an excellent student chooses to use. After a 4-5 year college career, a student who followed the distributed practice technique would be miles ahead of a student who followed the massed practice technique. Unfortunately, some college courses encourage massed practice by giving only 2-3 exams during the semester (and little else for assessment). When only 2-3 exams are given, the student masses study sessions immediately prior to each exam. This testing frequency (2-3 exams/semester) also promotes, the less desirable, rote learning.

How can a student implement distributed practice? Well, it takes motivation and determination to get this all rolling. Probably one good way is to schedule study times on a week to week basis at the beginning of each semester. That is, set aside one 50 minute study session each day for each course. Do this for Monday through Saturday, leaving Sunday as an off-day or catch-up day or even as a total relaxation day or family day. After the semester gets rolling, adjustments may need to be done. Perhaps some courses don't need the daily 50 minute study session M-Sat. with some sessions skipped during the week. In other cases, some courses may require more than one daily study session. Only the individual student can judge whether adjustments are needed. If a student needs so much study time that there isn't enough time in the day to schedule sessions, that student should consider dropping a course or two.

For distributed practice to be successful, the student must be able to follow his/her study schedule. \textbf{Don't let interruptions spoil it}. Think of your study schedule as a work schedule, something that must be followed. If you find that other people/other activities prevent you from keeping on schedule, then you are going to falter. Go hide someplace during your study sessions (the library works good for this if you find a corner up in the stacks). Another hint, take study breaks- study for 50 minutes then get up for a 5-10 minute break. Then come back to more of the same subject, or better yet, go on to a new subject. Another hint- try not to take two similar courses during the same semester. Sometimes when a student takes two similar courses, material from one class may interfere with the ability to learn material in the other class. This isn't always the case, but in general it is better to take courses (in any one semester) that are distinct from each other.
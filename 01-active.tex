\chapter{Active Learning}

This is a very wide and deep topic that I will only attempt to summarize here. The gist of the matter is this - \textbf{Learners Should Be Active}. Anything a learner can do to be actively involved in the subject at hand is the best plan of attack. The \textbf{best non-example} is this: Go to a lecture in one of your courses and sit passively all hour, then leave. This is the exact opposite of active learning. It could well be called passive learning. You just sit and listen, and perhaps take verbatim notes (writing down exactly what the instructor places on the overhead or blackboard). Couldn't a robot do that?


\marginnote{
I believe the Chinese have a saying that goes something like this:

\textit{Hear and Forget,}

\textit{See and Remember,}

\textit{Do and Understand.}}

Active learning is ``doing'' and this leads to understanding.

The \textbf{best example} of active learning is this: Go deliver a lecture. That requires a lot of preparation and understanding of the material. And after its all done, the subject is well understood by the individual who delivered the lecture (the audience may or may not have connected at all with what was said-in fact, a great lecture can be delivered to an empty room!). In most college classrooms however, the one (the student, the learner) who should be the most active is, in fact, the least active. It's kind of absurd that the one who knows the material best (the instructor) is the most active in our classrooms. This goes against the principle of active learning. Below I have listed some things that a student can do to promote active learning in and out of the classroom.


\section{Suggestions}

\begin{itemize}
	\item Don't take verbatim notes in a lecture course. As you actively listen in lecture, construct a set of notes that mean something to you. This may involve making drawings of material presented as text on the overhead. It may involve incomplete items that you will fill in after class after you have consulted the textbook or talked with the instructor. It may involve Mind Maps (see the book by Tony Buzan).
	
	\item Ask more questions in class!! Most students appear trained to come into a classroom and passively sit all hour. Break the mold. Ask questions that will help you make sense of the lecture material. Not only are you helping yourself, but also some other student who probably wanted to ask the same question. Questions from students who have kept up with the material (and attend class) are a valuable asset to the instructor because it is important feedback that lets the instructor know what was perhaps missed by students (or what needs further explanation). Questions from students who frequently miss class (or don't study) are a pain in the rear because their questions don't reflect the state of learning of the entire class. Be a good learner and your questions will automatically be good.
	
	\item Seek out other sources of the course material. Don't just rely on what the instructor gives you. Go find other texts, talk to other instructors, find references in the library, search the Internet, etc. Some of the material you find will be better than what is presented in your class.
	
	\item Study with a partner and take turns orally presenting topics to each other (with and without your notes as help).
	
	\item Create possible quiz/test questions and exchange them with someone in class who also made up questions.
	
	\item Place terms and their definitions on opposite sides of index cards for later review/self-quizzing.
	
	\item Obtain old quizzes/exams and try to answer them without any help.
\end{itemize}

\chapter{Generation Effect}

The Generation Effect refers to the fact that you will remember something better if you are involved in its creation or can create it after study. For example, in a course you may see the below table in a textbook or handout. During your study sessions you would usually look over the table and note its content. However, that is not all that should be done. You should attempt to reproduce the table from scratch. That is, can you reproduce the table starting with a blank sheet of paper? When you attempt that, it will be clear that you don't know the information in the table as well as your thought. When you study like this, you will remember so much more about the table information. 

\begin{table}[h]
	\begin{center}
		\footnotesize%
		\begin{tabular}{cc}
			\toprule
			Age of Bull (months) & Testes weight (g) \\
			\midrule
			2 & 20 \\
			12 & 370 \\
			36 & 590 \\
			\bottomrule
		\end{tabular}
	\end{center}
	\caption{Age of Bull vs Testes weight.}
	\label{tab:bull}
\end{table}

This same principle applies to figures and any text material. During study periods, you should attempt to reproduce important figures or terms from scratch. Take the word acetylcholine, for example. The best method to help learn its spelling is to write it down on paper and pronounce it. Take the word apart: acetyl-cho-line. Define it: acetylcholine is a neurotransmitter. Work with the word several times over several days. Then try to generate it.

It's one thing to say you know something, but until you can draw it (or speak it) without looking at the originial, you may well be fooling yourself. That leads to a good question-Who is on the U.S. quarter (coin) and which way is he facing? You probably will guess (although you have handled a quarter many times), but if you have drawn the coin several times you would remember better.
\chapter[Preview-View-Review]{Preview-View-Review System of Study}

This technique can be applied in many ways during the learning process. The View in this case refers to either a lecture, book chapter, lab handout, or anything to be studied. Most of us go to the lecture, read the chapter or lab handout without preparing our mind for what is to come during the lecture or reading. What would be best is to prepare our mind for the upcoming material. This is where the Preview comes in handy. The purpose of the Preview is to let our minds get a flavor of what is to come before the actual event occurs. In other words, preview the material before we study (or listen/take notes in the case of a lecture). During the process of Preview, our mind will automatically activate related knowledge it has stored on the subject. For example, if we were going to read a chapter on kidney function, the Preview would activate related knowledge and make it more likely that we would remember things during the View (for example, the actual reading of the chapter).

So, what actually is the Preview? Well, it can be many different things. Lets say we are about to read the chapter on kidney function. One technique (for a Preview) would be to skim the chapter reading bold text, looking at figures and tables, and reading questions/summary statements at the end of the chapter. This process would give us a flavor of things to come during the full reading of the chapter. Cognitive psychologists would call the Preview an advanced organizer. It is well known that if you preview something, you are more likely to remember it after the study session. In other words, you are more likely to store it in memory by attaching it to items previously learned. One other example: If you are going to a lecture on kidney function, a good Preview would be to discuss kidney function with a fellow student before getting to class. Between the two of you, many related topics would have been brought up and discussed before the lecture started. In this manner, your mind would have opened the file entitled ``kidneys'' and you would be prepared to put more items into the file during lecture.

Now, what about the Review part of this. The Review refers to a process of refreshing your mind of what has been encountered during the View. The Review is exactly how it sounds, to review. Much has been written on reviewing and I suggest you search the cognitive psychology literature for this information. Briefly, there is an art to reviewing. First, material should be reviewed more than a total of one time. Some would suggest the first Review  take place shortly after the View has occurred (within one hour). Then, the next (second) Review should take place one day after the View. The third Review should then occur one week after the original View. Now, depending upon the material, subsequent reviews would take place on a weekly to monthly basis. Any given Review should be short, say 5-10 minutes. The objective is to scan the material, activating memories as you go along.

In summary, this technique can be applied to almost any situation. Probably, for a beginner in this area, the first step to take is to practice previewing material. Obviously, discipline and motivation are required for success. Good Luck.
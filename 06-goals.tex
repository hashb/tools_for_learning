\chapter{Goals are Golden}

Setting goals is a very helpful technique that students can implement at any time. Setting goals and then reaching them can be, psychologically, very uplifting. It feels good to reach a goal. But there's also a danger. Not reaching a goal can be a downer or outright depressing. So be careful! Start with small, easy goals and then build them up to stretch your ability.

Here are some rules for setting goals. 

Goals should be \textbf{S. M. A. R. T.}

\textbf{S}pecific goals are the best. Good example of this: Read pages 12-21 in physics (chapter 1) tonight starting at 7pm. Here's the bad version: Read in my physics book sometime today.

\textbf{M}easurable goals are good. Good example of this: Read 20 pages in my chemistry book by noon today. Here's the bad version: Read some pages in my chemistry book this week.

\textbf{A}ttainable goals work best. Good example of this: Attend every class session this week and work at my job at Smitty's a total of 8 hours this week. Here's the bad version: Attend every class session this week and work 30 hours this week at my job at Smitty's and work 20 hours at Hooters.

\textbf{R}ealistic goals should be proposed. Good example of this: Since I want to improve my GPA (now standing at 1.81) I will try to get B's or better in all my classes. Here's the bad version: Since I want to improve my GPA (now standing at 1.81) I will try to get A's in all my classes.

\textbf{T}imely goals help. Good example of this: I am going to read 2 chapters in my biology book by noon tomorrow. Here's the bad version: I am going to read all my assigned reading in biology by the time of exam \#1 (which is scheduled 3 weeks from today).